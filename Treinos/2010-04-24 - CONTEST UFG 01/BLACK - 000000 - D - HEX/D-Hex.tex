% XeLaTeX can use any Mac OS X font. See the setromanfont command below.
% Input to XeLaTeX is full Unicode, so Unicode characters can be typed directly into the source.

% The next lines tell TeXShop to typeset with xelatex, and to open and save the source with Unicode encoding.

%!TEX TS-program = xelatex
%!TEX encoding = UTF-8 Unicode

\documentclass[14pt]{article}
\usepackage{geometry}                % See geometry.pdf to learn the layout options. There are lots.
\geometry{a4paper}                   % ... or a4paper or a5paper or ... 
%\geometry{landscape}                % Activate for for rotated page geometry
%\usepackage[parfill]{parskip}    % Activate to begin paragraphs with an empty line rather than an indent
\usepackage{graphicx}
\usepackage{amssymb}

% Will Robertson's fontspec.sty can be used to simplify font choices.
% To experiment, open /Applications/Font Book to examine the fonts provided on Mac OS X,
% and change "Hoefler Text" to any of these choices.

\usepackage{fontspec,xltxtra,xunicode}
\defaultfontfeatures{Mapping=tex-text}
\setromanfont[Mapping=tex-text]{Hoefler Text}
\setsansfont[Scale=MatchLowercase,Mapping=tex-text]{Gill Sans}
\setmonofont[Scale=MatchLowercase]{Andale Mono}

\title{Treino 1 - Maratona de Programa\c{c}\~ao }
\author{Universidade Federal de Goi\'as}
\date{24/04/2010}                                           % Activate to display a given date or no date

\begin{document}
\maketitle

% For many users, the previous commands will be enough.
% If you want to directly input Unicode, add an Input Menu or Keyboard to the menu bar 
% using the International Panel in System Preferences.
% Unicode must be typeset using a font containing the appropriate characters.
% Remove the comment signs below for examples.

% \newfontfamily{\A}{Geeza Pro}
% \newfontfamily{\H}[Scale=0.9]{Lucida Grande}
% \newfontfamily{\J}[Scale=0.85]{Osaka}

% Here are some multilingual Unicode fonts: this is Arabic text: {\A السلام عليكم}, this is Hebrew: {\H שלום}, 
% and here's some Japanese: {\J 今日は}.

\section{Problema D - Convers\~ao Hexadecimal}

\subsection{Descri\c{c}\~ao}

Durante um treinamento o t\'ecnico da equipe ``Noob Inside" ficou muito triste ao perceber que os membros da equipe n\~ao sabiam praticamente nada quando o assunto era convers\~ao de bases. Ap\'os uma bela aula sobre bases num\'ericas, ele passou alguns exerc\'icios para seus pupilos praticarem.

Os competidores da ``Noob Inside" devem escrever um programa capaz de converter um n\'umero na base Hexadecimal sem sinal para seu correspondente Octal ( base 8 ).

Nota: Um valor hexadecimal \'e uma maneira de representar n\'umeros na base $16$. Os digitos 0-9 continuam correspondendo aos valores 0-9, e s\~ao acrescentados os digitos A-F, com A correspondendo ao $10$, B ao $11$, etc.. ( F corresponde ao $15$ ).

Por exemplo, o n\'umero hexadeciam A$10$B corresponde ao decimal $ 10*16^3 + 1*16^2 + 0*16^1 + 11*16^0 = 41227$. O correspondente octal seria o  $120413$,  uma vez que $ 1*8^5 + 2*8^4 + 0*8^3 + 4*8^2 + 1*8^1 + 3*8^0 = 41227 $.

\subsection{Tarefa}

Sua tarefa \'e ajudar os membros da ``Noob Inside" que n\~ao est\~ao conseguindo resolver esse problema. Portanto, dado um n\'umero Hexadecimal de at\'e $100000$ digitos, escreva um programa capaz de convert\^e-lo para octal.

\subsection{Entrada}
A primeira linha da entrada cont\'em um inteiro T ( $0 < T \leqslant 50$\ ), especificando o n\'umero de casos de teste. Cada caso teste \'e composto por uma \'unica linha contendo um n\'umero na base hexadecimal de at\'e $100000$ digitos e sem zeros \`a esquerda. Ou seja, $1$AB ao inv\'es de $001$AB.


\subsection{Sa\'{\i}da}
A sa\'{\i}da para cada caso teste ser\'a uma \'unica linha contendo o valor octal correspondente a entrada, sem zeros a esquerda. Se a entrada for $0$ a sa\'ida deve ser $0$.

\subsection{Exemplo}
\subsubsection{Entrada}
\begin{verbatim}
2
A10B
123ABC
 \end{verbatim}
\subsubsection{Sa\'{\i}da}
 \begin{verbatim}
120413
4435274
 \end{verbatim}


\end{document}  