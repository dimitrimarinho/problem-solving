% XeLaTeX can use any Mac OS X font. See the setromanfont command below.
% Input to XeLaTeX is full Unicode, so Unicode characters can be typed directly into the source.

% The next lines tell TeXShop to typeset with xelatex, and to open and save the source with Unicode encoding.

%!TEX TS-program = xelatex
%!TEX encoding = UTF-8 Unicode

\documentclass[14pt]{article}
\usepackage{geometry}                % See geometry.pdf to learn the layout options. There are lots.
\geometry{a4paper}                   % ... or a4paper or a5paper or ... 
%\geometry{landscape}                % Activate for for rotated page geometry
%\usepackage[parfill]{parskip}    % Activate to begin paragraphs with an empty line rather than an indent
\usepackage{graphicx}
\usepackage{amssymb}

% Will Robertson's fontspec.sty can be used to simplify font choices.
% To experiment, open /Applications/Font Book to examine the fonts provided on Mac OS X,
% and change "Hoefler Text" to any of these choices.

\usepackage{fontspec,xltxtra,xunicode}
\defaultfontfeatures{Mapping=tex-text}
\setromanfont[Mapping=tex-text]{Hoefler Text}
\setsansfont[Scale=MatchLowercase,Mapping=tex-text]{Gill Sans}
\setmonofont[Scale=MatchLowercase]{Andale Mono}

\title{Treino 1 - Maratona de Programa\c{c}\~ao }
\author{Universidade Federal de Goi\'as}
\date{24/04/2010}                                           % Activate to display a given date or no date

\begin{document}
\maketitle

% For many users, the previous commands will be enough.
% If you want to directly input Unicode, add an Input Menu or Keyboard to the menu bar 
% using the International Panel in System Preferences.
% Unicode must be typeset using a font containing the appropriate characters.
% Remove the comment signs below for examples.

% \newfontfamily{\A}{Geeza Pro}
% \newfontfamily{\H}[Scale=0.9]{Lucida Grande}
% \newfontfamily{\J}[Scale=0.85]{Osaka}

% Here are some multilingual Unicode fonts: this is Arabic text: {\A السلام عليكم}, this is Hebrew: {\H שלום}, 
% and here's some Japanese: {\J 今日は}.

\section{Problema A - GSM}

\subsection{Descri\c{c}\~ao}

A parte mais importante de uma rede GSM \'e a chamada BTS ( \emph{ Base Transceiver Station} ). Esses transceivers formam zonas chamas c\'elulas (\ termo que deu origem ao nome \emph{telefone celular}\ ) e cada celular se conecta a BTS com o sinal mais forte ( em uma vis\~ao simplificada ). \'E claro que BTSes precisam de bastante aten\c{c}\~ao e t\'ecnicos precisam checar seu funcionamento periodicamente.

T\'ecnicos da ACM se depararam com um problema interessante recentemente. Dado um conjunto de BTSes a visitar, eles precisam encontrar o caminho mais curto para visitar todos os pontos dados e retornar \`a Central. Alguns programadores gastaram v\'arios meses estudandodo esse problema mas n\~ao obtiveram resultados satisfat\'orios. At\'e que um dos programadores encontrou refer\^encis a esse problema em um artigo. Ele descobriu que esse \'e na verdade um problema a muito conhecido, chamado \emph{ Travelling Salesman Problem } - TSP ( Problema do Caixeiro viajante. ) e \'e muito dif\'icil de se resolver. Se tivermos N BTSes \`a visitar,  podemos visit\'a-las em qualquer ordem, tendo assim N! possibilidades pra examinar. A fun\c{c}\~ao que expressa esse n\'umero \'e chamada fatorial, e pode ser computada como o produto $1*2*3*4*...*N$, e esse valor \'e muito grande mesmo para um N relativamente pequeno.

Os programadores se deram conta de que eles jamais conseguiriam resolver o problema, mas como eles j\'a receberam uma verba de pesquisa do governo, eles precisam continuar com os estudos e produzir algum resultado. Portanto, eles come\c{c}aram a estudar o comportamento da fun\c{c}\~ao fatorial.

Eles definiram uma fun\c{c}\~ao Z. Para todo inteiro positivo N,  Z(N) \'e o n\'umero de zeros no final da representa\c{c}\~ao decimal do n\'umero N!
Eles notaram que essa fun\c{c}\~ao nunca decresce. Se n\'os temos dois n\'umeros $N1 < N2$, $ Z(N1) \leq Z(N2)$. Isso pois \'e imposs\'ivel perder um zero a direita multiplicando-o por qualquer inteiro positivo. Os programadores acharam o comportamento da fun\c{c}\~ao Z bem interessante, e agora est\~ao desenvolvendo um programa de computador capaz de calcular o seu valor de forma eficiente.


\subsection{Tarefa}
Sua tarefa nesse problema \'e ajudar os programadores da ACM e escrever um uma vers\~ao Beta do programa que eles est\~ao desenvolvendo, que calcule o valor de Z(N) de forma eficiente.

\subsection{Entrada}
A entrada \'e composta por diversos casos teste, cada caso teste corresponde a um inteiro N (\ $0 < N \leq 10^{12} $\ ). A entrada deve ser lida at\'e o final do arquivo ( EOF ).


\subsection{Sa\'{\i}da}
A sa\'{\i}da para cada caso teste ser\'a uma \'unica linha contendo o valor Z(N).

\subsection{Exemplo}
\subsubsection{Entrada}
\begin{verbatim}
3
60
100
1024
23456
8735373
 \end{verbatim}
\subsubsection{Sa\'{\i}da}
 \begin{verbatim}
0
14
24
253
5861
2183837
 \end{verbatim}


\end{document}  